\chapter{Case Studies}

I have chosen two electives to fulfill my 12 credit points of elective coursework: \textit{ELEC5305: Acoustics, Speech and Signal Processing}, and \textit{ELEC5307: Advanced Signal Processing with Deep Learning}. Below I have outlined how I have met the learning outcomes from each unit of study.

\section{ELEC5305: Acoustics, Speech and Signal Processing}

\begin{itemize}
    \item LO1: Mastery of analytical and mathematical skills related to acoustic signal processing
    \begin{itemize}
        \item My thesis began with an in-depth exploration of foundational acoustic signal processing techniques such as the short-time Fourier transform, Mel spectrum, LOFAR spectra, and the Constant Q transform. These methods form the backbone of state-of-the-art approaches in underwater acoustic target recognition. By thoroughly understanding these techniques, I was able to analyse their suitability as feature inputs and build upon them to improve underwater target classification systems.
    \end{itemize}
    \item LO2: Proficiency in developing signal processing software
    \begin{itemize}
        \item Using MATLAB's Audio Processing Toolkit I developed algorithms to tackle a variety of acoustic signal processing challenges. For example, I implemented the $\ell_1$ detrending algorithm for spectral data in Chapter \ref{chap:detrending} to better understand the impact of preprocessing methods on classification accuracy.
    \end{itemize}
    \item LO3: Planning, designing, and reviewing signal processing systems
    \begin{itemize}
        \item As part of my work, I designed an end-to-end signal processing pipeline tailored for UATR tasks. This pipeline includes the selection of appropriate signal representations, feature extraction techniques, and integration with machine learning models for classification. Each stage of the pipeline is reviewed and evaluated based on its impact on overall system performance.
    \end{itemize}
    \item LO4: Developing innovative ideas in signal processing systems
    \begin{itemize}
        \item My thesis incorporates cutting-edge techniques such as self-supervised and unsupervised learning to complement traditional signal processing methods. For instance, I applied the Noise2Noise framework to underwater acoustic spectrograms and explored masking-based denoising techniques in Chapter \ref{chap:denoising}.
    \end{itemize}
    \item LO5: Communicating signal processing practice effectively
    \begin{itemize}
        \item I communicated the methodologies and findings of my work in a clear and accessible manner through informative visualisations and comprehensive documentation. These materials were prepared for both academic audiences and industry stakeholders, ensuring the applicability and relevance of my work.
    \end{itemize}
    \item LO6: Contributing to team-based projects
    \begin{itemize}
        \item As an intern with the Underwater Systems team at Thales, I worked collaboratively with colleagues, presenting weekly updates and incorporating their feedback to refine the project's direction. This experience not only strengthened my technical skills but also underscored the importance of teamwork in advancing research and development goals.
    \end{itemize}
\end{itemize}

\section{ELEC5307: Advanced Signal Processing with Deep Learning}

\begin{itemize}
    \item LO1: Using appropriate software platforms for multi-dimensional signal processing tasks
    \begin{itemize}
        \item My thesis integrates Python libraries such as TensorFlow, Keras, and Scikit-learn for implementing and training deep learning models. These are complemented by MATLAB, which I used for preprocessing tasks such as spectrogram generation and signal transformation. This combination allowed me to bridge traditional signal processing techniques with advanced deep learning methods effectively.
    \end{itemize}
    \item LO2: Applying machine learning and deep learning methods to multi-dimensional signal processing
    \begin{itemize}
        \item Throughout my research, I designed and implemented a variety of deep learning architectures to address multi-dimensional signal processing challenges in underwater acoustic target recognition. These include convolutional autoencoders for feature extraction, CNNs and U-Nets for image-based tasks such as denoising and segmentation, and hybrid models like CNN-LSTM to capture both spatial and temporal dependencies in spectrogram data.
    \end{itemize}
    \item LO3: Using existing machine learning and deep learning toolboxes
    \begin{itemize}
        \item My project extensively leveraged pre-built libraries and toolboxes to implement complex models and streamline experimentation. In Python, TensorFlow and Keras facilitated the construction and training of deep learning architectures, while MATLAB was used for signal preprocessing and transformation. This allowed me to focus on adapting and optimising these tools for specific UATR tasks, ensuring their applicability in real-world scenarios.
    \end{itemize}
    \item LO4: Reporting results professionally
    \begin{itemize}
        \item The outcomes of my research were communicated through structured presentations and this thesis. These documents adhere to academic and professional standards, incorporating clear visualisations, well-structured discussions, and actionable conclusions to ensure clarity and impact for a diverse audience of academic researchers and industry stakeholders.
    \end{itemize}
\end{itemize}